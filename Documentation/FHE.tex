\mainsection{FHE Security}\label{sec:fhe}
In this chapter we detail how FHE parameters are selected.
The first part deals with how the basic ``sizes'' are derived,
and the second part gives the mathematical justification (i.e.
noise analysis) for the equations used in the code for setting
up parameters.
Before proceeding it is perhaps worth detailing how
different security parameters are used within the code.

\begin{itemize}
\item $\Sacsecp$ (default $80$): 
This defined how many sacrifices we do per triple, 
to get the precise number you compute $\ceil{\Sacsecp/\log_2 p}$.
Thus this statistical security parameter is effectively
$\log_2 p$ unless $p$ is very small.
\item $\Macsecp$ (default $80$):
For the full threshold case this defined how many MACs are held
per data item, again to get the precise number you compute 
$\ceil{\Macsecp/\log_2 p}$.
Thus, again, this statistical security parameter is effectively
$\log_2 p$ unless $p$ is very small.
\item $\Soundsecp$ (default $128$):
This defines the soundness error of the ZKPoKs below,
the value of $2^{-\Soundsecp}$ defines the probability
that an adversary can cheat in a single ZKPoK.
\item $\ZKsecp$ (default $80$): 
This defines the statistical distance between the 
coefficients of the ring elements in an honest ZKPoK
transcript and one produced by a simulation.
\item $\DDsecp$ (default $80$):
This defines the statistical distance between the 
coefficients of the ring elements in the distribution produced
in the distributed decryption protocol below to the uniform
distribution.
\item \verb+comp_sec+ (default $128$):
This defines the computational security parameter for the
FHE scheme. 
\item $\epsilon$ (default $55$):
This defines the noise bounds for the FHE scheme below in
the following sense.
A FHE ciphertext in the protocol is gauranteed to decrypt
correctly, even under adversarial inputs assuming the
ZKPoKs are applied correctly, with probability $1-2^{-\epsilon}$.
In fact this is an under-estimate of the probability.
From $\epsilon$ we define $e_i$ such that 
$\mathsf{erfc}(e_i)^i \approx 2^{-\epsilon}$ and then we set $\gc_i = e_i^i$.
\end{itemize}
All of these parameters can be tweaked in the file
\verb+config.h+.
Another two parameters related to the FHE scheme are
\verb+hwt+, which is defined in \verb+Setup.cpp+ to
be equal to $h=64$.
This defines the number of non-zero coefficients in the
FHE secret key.
Another parameter is $\sigma$ which is the standard deviation
for our approximate discrete Gaussians, which we
hardwire to be $3.16 = \sqrt{10}$ (as we use the NewHope method
of producing approximate Gaussians).

\subsection{Main Security Parameters}
Our Ring-LWE samples are (essentially) always from an
approximate Gaussian distribution with standard deviation $3.16$
and from a ring with a two-power degree of $N$.
This is not strictly true as some noise samples come
from small Hamming Weight distributions, and some come
from distributions over $\{-1,0,1\}$. 
But the above are the main parameters, and have been used in prior works to
estimate Ring-LWE security in SHE settings.

Given these settings we need to determine the maximum 
(ciphertext) ring modulus $q$ that still gives us security.
For this we use Martin Albrecht's estimator which can
be found at
\begin{center}
\url{https://bitbucket.org/malb/lwe-estimator}
\end{center}
For various values of $n$ and (symmetric equivalent)
security levels \verb+comp_sec+ we then find the minimum secure
value of $q$.
This is done by running the following \verb+sage+ code
\begin{verbatim}
load("estimator.py")
n = 1024
comp_sec = 128
for i in range(10, 500, 5):
   q= 2^i
   costs = estimate_lwe(n, 3.16*sqrt(2*pi)/q, q, reduction_cost_model=BKZ.sieve,  \
                        skip=["arora-gb", "bkw", "dec", "mitm"])
   if any([cost.values()[0]<2^comp_sec for cost in costs.values()]):
       break
print i-5
\end{verbatim}
This will print a bunch of rubbish and then the number $29$. 
This means any $q < 2^{29}$ will be ``secure'' by the above definition of secure.

We did this in Feb 2018 and obtained the following table of values, giving maximum
values of $q$ in the form of $2^x$ for the value $x$ from the following table.
\begin{center}
\begin{tabular}{|c|c|c|c|}
\hline
$N$   & \verb+comp_sec+=80 & \verb+comp_sec+=128 & \verb+comp_sec+=256 \\
\hline
1024  &  44 &    29        & 16 \\
2048  &  86 &    56        & 31 \\
4096  & 171 &   111        & 60 \\
8192  & 344 &   220        & 120 \\
16384 & 690 &   440        & 239 \\
32768 & 998 &   883        & 478 \\
\hline
\end{tabular}
\end{center}
Any updates to this table needs to be duplicated in 
the file \verb+FHE_Params.cpp+ in the code base.

\subsection{Distributions and Norms}
Given an element $a \in R = \Z[X]/(X^{N}+1)$ (represented as a polynomial)
where $X^N+1$ is the $m=2 \cdot N$-th cyclotomic polynomial
(recall $N$ is always a power of two).
We define $\norm{a}_p$ to be the standard $p$-norm
of the coefficient vector (usually for $p=1, 2$ or $\infty$).
We also define $\norm{a}_p^\can$ to be the $p$-norm 
of the same element when mapped into the canonical embedding
i.e.
\[  \norm{a}_p^\can = \norm{\kappa(a)}_p \]
where $\kappa(a): R \longrightarrow \C^{\phi(m)}$ is the
canonical embedding.
The key three relationships are that 
\[ \norm{a}_\infty \le c_m \cdot \norm{a}_\infty^\can \quad
   \mbox{  and  } \quad
   \norm{a}_\infty^\can \le \norm{a}_1  \quad
   \mbox{  and  } \quad
   \norm{a}_1 \le \phi(m) \cdot \norm{a}_\infty.
\]
for some constant $c_m$ depending on $m$.
Since in our protocol we select $m$ to be a power of two, we have $c_m=1$.
In particular (which will be important later in measuring the
noise of potentially dishonestly generated ciphertexts) we have
\[ \norm{a}_{\infty}^{\can} \le \phi(m) \cdot \norm{a}_{\infty}. \]
We also define the \emph{canonical embedding norm reduced modulo~$q$} 
of an element $a\in R$ as the smallest canonical embedding norm of any
$a'$ which is congruent to $a$ modulo~$q$. We denote it as
\[
|a|_q^{can} = \min
  \{~\norm{a'}_\infty^\can :~ a'\in R,~ a' \equiv a \pmod{q}~\}.
\]
We sometimes also denote the polynomial where the minimum is obtained
by $[a]_q^{\can}$, and call it the {\em canonical reduction} of
$a$ modulo $q$. 

Following \cite{GHS12c}[Appendix A.5] we examine the variances
of the different distributions utilized in our protocol.
\begin{itemize}
\item $\HWT(h,N)$: This generates a vector of length $N$
      with elements chosen at random from $\{-1,0,1\}$ subject to
      the condition that the number of non-zero elements is equal to $h$.
\item $\ZO(0.5,N)$:  This generates a vector of length $N$
      with elements chosen from $\{-1,0$, $1\}$ such that the
      probability of coefficient is $p_{-1}=1/4$, $p_0=1/2$
      and $p_1=1/4$.
\item $\dN(\sigma^2,N)$: This generates a vector of
      length $N$ with elements chosen according to an approximation to
      the discrete Gaussian distribution with variance $\sigma^2$.
\item $\RC(0.5,\sigma^2,N)$: This generates a triple of
      elements $(v,e_0,e_1)$ where $v$ is sampled from $\ZO_s(0.5,N)$
      and $e_0$ and $e_1$ are sampled from $\dN_s(\sigma^2,N)$.
\item $\calU(q,N)$: This generates a vector of length $N$ 
      with elements generated uniformly modulo $q$.
\end{itemize}
Let $\zeta_m$ denote any complex primitive $m$-th root of unity.
Sampling $a \in R$ from $\HWT(h,\phi(m))$ and looking at $a(\zeta_m)$
produces a random variable with variance $h$, when sampled
from $\ZO(0.5,\phi(m))$ we obtain variance $\phi(m)/2$,
when sampled from $\dN(\sigma^2,\phi(m))$ we obtain variance
$\sigma^2 \cdot \phi(m)$ and when sampled from $\calU(q,\phi(m))$
we obtain variance $q^2 \cdot \phi(m)/12$.
By the law of large numbers we can use $\gc_1 \cdot \sqrt{V}$,
where $V$ is the above variance, as a high probability bound
on the size of $a(\zeta_m)$ (the probability is $1-2^{-\epsilon}$, 
and this provides a bound on the canonical embedding norm of $a$.

If we take a product of $t$ such
elements with variances $V_1, V_2, \ldots, V_t$
then we use $\gc_t \cdot \sqrt{V_1 \cdot V_2 \cdots V_t}$
as the resulting bounds.
In our implementation we approximate $\dN(\sigma^2,n)$ using
the binomial method from the NewHope paper, with a standard
deviation of $3.16 = \sqrt{10}$. In particular this means
any vector sampled from $\dN(\sigma^2,n)$ will have
infinity norm bounded by $20$.


\subsection{The FHE Scheme and Noise Analysis}
Note that whilst the scheme is ``standard'' some of the
analysis is slightly different from that presented in the \cite{SPDZ2} paper 
and that presented in the \cite{GHS12c} paper.
We use the notation of the \cite{SPDZ2} paper in this section,
and assume the reader is familiar with prior work.
The key differences are:
\begin{itemize}
\item Unlike in SPDZ-2 we assume a perfect distributed key generation method
amongst the $n$ players.
\item Unlike in SPDZ-2 we are going to use an actively secure ZKPoK (see later),
which will make the actual noise analysis much more complex.
\end{itemize}


\subsubsection{Key Generation:}
The main distinction between the assumptions here and those used in 
the \cite{SPDZ2} paper, is that in the latter a specific distributed 
key generation protocol for the underlying threshold FHE keys was
assumed. In SCALE we assume a `magic black box' which distributes
these keys, which we leave to the application developer to create.
The secret key $\sk$ is selected from a distribution with
Hamming weight $h$, i.e. $\HWT(h,\phi(m))$, 
and then it is distributed amongst the $n$ parties by simply producing a random 
linear combination, and assigning each party one of the sums.
The switching key data is produced in the standard way, i.e.
in a non-distributed trusted manner.
We assume a two-leveled scheme with moduli $p_0$ and $p_1$ with $q_1=p_0 \cdot p_1$
and $q_0=p_0$.
We require 
\begin{align*}
   p_1 & \equiv 1 \pmod{p}, \\
   p_0 - 1 & \equiv p_1-1 \equiv p-1 \equiv 0 \pmod{\phi(m)}.
\end{align*}
In particular the public key is of the form $(a,b)$ where
\[ a \asn \calU(q,\phi(m)) \quad \mbox{ and } \quad b = a \cdot \sk + p \cdot \epsilon \]
where $\epsilon \asn \dN(\sigma^2,\phi(m))$ 
and the switching key data $(a_{\sk,\sk^2},b_{\sk,\sk^2})$ is of the form
\[ a_{\sk,\sk^2} \asn \calU(q,\phi(m)) \quad \mbox{ and } \quad 
   b_{\sk,\sk^2} =   a_{\sk,\sk^2} \cdot \sk + p \cdot e_{\sk,\sk^2} - p_1 \cdot \sk^2 \]
where $e_{\sk,\sk^2} \asn \dN(\sigma^2,\phi(m))$.
We take $\sigma=3.16$ as described above in what follows.

\subsubsection{Encryption:}
To encrypt an element $m\in R$, we choose $v, e_0, e_1 \asn \RC(0.5,\sigma^2,n)$, i.e.
\[
v \asn \ZO(0.5, \phi(m)) ~~~\mbox{and}~~ e_0,e_1 \asn \dN(\sigma^2,\phi(m)) 
\]
Then we set $c_0 = b \cdot v + p \cdot e_0+m$,~ $c_1=a\cdot v+p\cdot
e_1$, and set the initial ciphertext as $\ct'=(c_0,c_1)$.
We calculate a bound (with high probability) on the output noise of
an honestly generated ciphertext to be
\begin{align*}
  \norm{c_0 - \sk \cdot c_1}_\infty^\can 
     &= \norm{((a \cdot \sk+ p \cdot \epsilon) \cdot v +p \cdot e_0 +m
                - (a \cdot v+ p \cdot e_1) \cdot \sk }_\infty^\can \\
     &= \norm{m + p \cdot (\epsilon \cdot v +e_0 - e_1 \cdot \sk)}_\infty^\can \\
     &\le \norm{m}_\infty^\can 
         +p \cdot \left( \norm{\epsilon\cdot v}_\infty^\can
                       + \norm{e_0}_\infty^\can
                       + \norm{e_1 \cdot \sk}_\infty^\can
                  \right) \\
     &\le \phi(m) \cdot p/2
	  + p \cdot \sigma \cdot 
             \left( \gc_2 \cdot \phi(m) / \sqrt{2}
			  + \gc_1 \cdot \sqrt{\phi(m)}
			  + \gc_2 \cdot \sqrt{h \cdot \phi(m)}
		    \right) = B_\clean.
\end{align*}
Note this is a probablistic bound and not an absolute bound.

\vspace{5mm}

\noindent
However, below we will only be able to guarantee the
$m, v, e_0$ and $e_1$ values of a {\bf sum} of $n$ fresh ciphertexts
(one from each party) are selected subject to
\[ \norm{v}_\infty \le 2^{\ZKsecp + U/2 + 2} \cdot n \quad 
    \mbox{  and  } \quad
   \norm{e_0}_\infty, \norm{e_1}_\infty \le 20 \cdot 2^{\ZKsecp + U/2 + 2} \cdot n \quad
    \mbox{  and  } \quad
   \norm{m}_\infty \le 2^{\ZKsecp/2 + U/2 + 1} \cdot n \cdot p,
\]
where $\ZKsecp$ is our soundness parameter for the 
zero-knowledge proofs
and $U$ is selected so that $(2 \cdot \phi(m))^U > 2^\Soundsecp$.
In this situation we obtain the bound, using the inequality above
between the infinity norm in the polynomial embedding
and the infinity norm in the canonical embedding,
\begin{align*}
  \norm{c_0 - \sk \cdot c_1}_\infty^\can
  &\le \sum_{i=1}^n
  	\norm{2 \cdot m_i}_\infty^\can
	+ p \cdot \Big(\norm{\epsilon}_\infty^\can \cdot \norm{2 \cdot e_{2,i}}_\infty^\can 
	+ \norm{2 \cdot e_{0,i}}_\infty^\can \\
  &\hspace{3.5cm}
	+ \norm{\sk}_\infty^\can \cdot \norm{2 \cdot e_{1,i}}_\infty^\can \Big) \\
   &\le 2 \cdot \phi(m) \cdot 2^{\ZKsecp+U/2+1} \cdot n \cdot p/2  \\
   & \hspace{2cm}
	+ p \cdot \Big(
	 \gc_1 \cdot \sigma \cdot \phi(m)^{3/2} \cdot 2 \cdot 2^{\ZKsecp+U/2+1} \cdot n\\
  &\hspace{3cm}
	  + \phi(m) \cdot 2 \cdot  2^{\ZKsecp+U/2+1} \cdot n \cdot 20  \\
  &\hspace{3cm}
	  +  \gc_1 \cdot \sqrt{h} \cdot \phi(m) \cdot 2 \cdot 2^{\ZKsecp+U/2+1} \cdot n \cdot 20 
	\Big)  \\
  &=\phi(m) \cdot 2^{\ZKsecp+U/2+2} \cdot n \cdot p
	\cdot \Big( \frac{41}{2} + \gc_1 \cdot \sigma \cdot \phi(m)^{1/2}  
	           +  20 \cdot \gc_1 \cdot \sqrt{h} 
	\Big)   \\
  & = B_\clean^\dishonest.
\end{align*}
Again this is a probabilistic bound (assuming validly
distributed key generation), but assumes the worst case
ciphertext bounds.

\subsubsection{$\SwitchModulus((c_0,c_1))$:}
This takes as input a ciphertext modulo $q_1$ and outputs a ciphertext mod $q_0$.
The initial ciphertext is at level $q_1=p_0 \cdot p_1$, with $q_0=p_0$.
If the input ciphertext has noise bounded by $\nu$
in the canonical embedding
then the output ciphertext will have noise bounded by $\nu'$ in
the canonical embedding, where
\[ \nu' = \frac{\nu}{p_1} + B_\scale. \]
The value $B_\scale$ is an upper bound on the quantity
$\norm{\tau_0+\tau_1 \cdot \sk}_\infty^\can$, where
$\kappa(\tau_i)$ is drawn from a distribution
which is close to a complex Gaussian with variance $\phi(m)\cdot p^2/12$.
We therefore, we can (with high probability) take the upper
bound to be
\begin{align*}
    B_\scale 
	&= c_1 \cdot p \cdot \sqrt{ \phi(m)/12}
	+  c_2 \cdot p \cdot \sqrt{ \phi(m) \cdot h/12}, \\
\end{align*}
This is again a probabilistic analysis, assuming validly generated
public keys.

\subsubsection{$\Dec_{\sk}(\ct)$:}
As explained in \cite{SPDZ2,GHS12c} this procedure works when the noise 
$\nu$ (in the canonical embedding) associated with a ciphertext satisfies $c_m \cdot \nu  < q_{0}/2$.
However, as we always take power of two cyclotomics we have $c_m=1$

\subsubsection{$\DistDec_{\{\sk_i\}}(\ct)$:}
There are two possible distributed decryption protocols.
The first one is from \cite{SPDZ} is for when we want to obtain a
resharing of an encrypted value, along with a fresh ciphertext.
The second version is from \cite{KPR}, where we do not need to
obtain a fresh encryption of the value being reshared.


\begin{Boxfig}{The sub-protocol for additively secret sharing a plaintext $\vm \in (\F_{p^k})^s$ on input a ciphertext $e_\vm=\Enc_\pk(\vm)$.}{reshare}{$\mathsf{Reshare-1}$}
Input is $e_\vm$, where $e_\vm=\Enc_\pk(\vm)$ is a public ciphertext.

Output is a share $\vm_i$ of $\vm$ to each player $P_i$; and a ciphertext $e'_\vm$. 

The idea is that $e_\vm$ could be a product of two ciphertexts, which
$\mathsf{Reshare}$ converts to a ``fresh'' ciphertext $e'_\vm$. Since $\mathsf{Reshare}$ uses distributed decryption (that may return an incorrect result), it is not guaranteed that $e_\vm$ and $e'_\vm$ contain the same value, but it
{\em is} guaranteed that $\sum_i \vm_i$ is the value contained in $e'_\vm$.
\begin{enumerate}
\item Each player $P_i$ samples a uniform $\vf_i\in (\F_p)^N$. Define $\vf:=\sum_{i=1}^{n}\vf_i$.
\item Each player $P_i$ computes and broadcasts $e_{\vf_i}\asn\Enc_\pk(\vf_i)$.\label{reshare:enc}
\item Each player $P_i$ runs the proof below as a prover on $e_{\vf_i}$. The protocol aborts if any proof fails
and if successful each player obtains $e_\vf\asn e_{\vf_1}\boxplus\dots\boxplus e_{\vf_n}$.
\item The players compute $e_{\vm+\vf}\asn e_\vm\boxplus e_\vf$.
\item The players decrypt $e_{\vm+\vf}$ as follows:
\begin{enumerate}
\item Player one computes $\vv_1 = e_{\vm+\vf}^{(0)}-\sk_1 \cdot e_{\vm+\vf}^{(1)}$ and player $i \ne 1$ computes.
$\vv_i = -\sk_i \cdot e_{\vm+\vf}^{(1)}$.
\item All players compute $\vt_i = \vv_i + p \cdot \vr_i$ for some random element
with infinity norm given by $2^\DDsecp \cdot B/p$, where $\DDsecp$ is the parameter defining statistical distance for the distributed decryption.
\item The parties broadcast $\vt_i$.
\item The parties compute $\vm+\vf = \sum \vt_i \pmod{p}$.
\end{enumerate}
\item $P_1$ sets $\vm_1\asn \vm+\vf-\vf_1$, and
each player $P_i$ ($i\neq 1$) sets $\vm_i\asn -\vf_i$.
%\item Each player $P_i$ now holds a share $m_i$ of the secret value $m$.
\item All players set $e'_\vm \asn \Enc_{pk}(\vm+\vf)\boxminus e_{\vf}$,
where a default value for the randomness is used when computing $\Enc_{pk}(\vm+\vf)$.
\end{enumerate}
\end{Boxfig}


\paragraph{Reshare Version 1:}
This is described in \figref{reshare}.
The value $B$ in the protocol is an upper bound on the noise in the canonical embedding
$\nu$ associated with a ciphertext we will decrypt in our protocols.
To ensure valid distributed decryption we require
\[ 2 \cdot (1+n \cdot 2^\DDsecp) \cdot B < q_{0}. \]

Given a value of $B$, we therefore will obtain a lower bound
on $p_0$ by the above inequality.
The addition of a random term with infinity norm bounded by
$2^\DDsecp \cdot B/p$ in the distributed decryption procedure
ensures that the individual {\em coefficients} of the sum
$\vt_1+\cdots+\vt_n$ are statistically indistinguishable from
random, with probability $2^{-\DDsecp}$.
This does not imply that the adversary has this probability of
distinguishing the simulated execution in \cite{SPDZ} from the
real execution; since each run consists of the exchange of 
$\phi(m)$ coefficients, and the protocol is executed many times
over the execution of the whole protocol.
We however feel that setting concentrating solely on the
statistical indistinguishability of the coefficients is valid 
in a practical context.

\paragraph{Reshare Version 2:}
This is given in \figref{distdec}.
This protocol is simpler as it does not require the resharing to a 
new ciphertext.
So in particular players do not need to provide encryptions of
their $\vf_i$ values, and hence there is no need
to execute the ZKPoK needed in the previous protocol.

\begin{Boxfig}{Distributed decryption to secret
    sharing}{distdec}{$\mathsf{Reshare-2}$}
Furthermore, let $B$ denote a bound on the noise, that is $\norm{c_0 - s \cdot c_1}_\infty$.
\begin{enumerate}
\item Party $i$ samples $\vf_i \asn [0,B\cdot 2^\DDsecp]^N$.
\item Party $1$ computes
  $\vv_i := (c_0 - s_1 \cdot c_1) - \vf_1 \bmod q$, and every other
  party $i$ computes $\vv_i :=  -s_i \cdot c_1- \vf_i$.
\item Party $i$ broadcasts $\vv_i$.
\item Party 1 outputs
  $\vm_1 := (\sum_{i=1}^n \vv_i \bmod q + \vf_1) \bmod p$, and
  every other party $i$ outputs $\vm_i := \vf_i \bmod p$.
\end{enumerate}
\end{Boxfig}





\subsubsection{$\SwitchKey(d_0,d_1,d_2)$:}
In order to estimate the size of the output noise term 
in the canonical embedding we need first to estimate the size of the term
(again probabilistically assuming validly generated public
keys)
\[ \norm{p \cdot d_2 \cdot \epsilon_{\sk,\sk^2}}_\infty^\can. \]
Following \cite{GHS12c} we assume heuristically that $d_2$
behaves like a uniform polynomial with coefficients drawn from
$[-q_0/2,\ldots,q_0/2]$ (remember we apply $\SwitchKey$ at level
zero).
So we expect
\begin{align*}
 \norm{ p \cdot d_2 \cdot e_{\sk,\sk^2}}_\infty^\can 
     &\le p \cdot c_2 \cdot \sqrt{q_0^2/12 \cdot \sigma^2 \cdot \phi(m)^2} \\
     & = p \cdot c_2 \cdot q_0 \cdot \sigma \cdot \phi(m)/\sqrt{12} \\
     & = B_\KS \cdot q_0.
\end{align*}
Thus
\[ B_\KS = p \cdot c_2 \cdot \sigma \cdot \phi(m)/\sqrt{12}. \]
Then if the input to $\SwitchKey$ has noise bounded by $\nu$ then the output 
noise value in the canonical embedding will be bounded by
\[ \nu+\frac{B_\KS \cdot p_0}{p_1} + B_\scale. \]


\subsubsection{$\Mult(\ct,\ct')$:}
Combining the all the above, if we take two ciphertexts of level one
with input noise in the canonical embedding bounded by $\nu$ and $\nu'$, the output noise level 
from multiplication will be bounded by
\[ \nu'' =      \left( \frac{\nu}{p_1} + B_\scale \right)
	  \cdot 
                \left( \frac{\nu'}{p_1} + B_\scale \right)
		+ \frac{B_\KS \cdot p_0}{p_1}+B_\scale.
\]

\subsubsection{Application to the Offline Phase:}
In all of our protocols the most complext circuit we will be evaluating 
is the following one: 
We first add $n$ ciphertexts together and perform a multiplication, giving a 
ciphertext with respect to modulus $p_0$ with noise in the canonical
embedding bounded by
\[
  U_1 = \left( \frac{B_\clean^\dishonest}{p_1}+B_\scale \right)^2
		+ \frac{B_{\KS} \cdot p_0}{p_1} + B_\scale.
\]
Recall that $B_\clean^\dishonest$ is the bound for a sum of 
$n$ ciphertexts, one from each party.
We then add on another $n$ ciphertexts, which are added at 
level one and then reduced to level zero.
We therefore obtain a final upper bound on the noise 
in the canonical embedding for our adversarially generated ciphertexts of
\[ U_2 = U_1 +  \frac{B_\clean^\dishonest}{p_1} + B_\scale. \]
To ensure valid (distributed) decryption, we require
\[ 2 \cdot U_2 \cdot (1+n \cdot 2^\DDsecp) < p_0, \]
i.e. we take $B=U_2$ in our distributed decryption protocol.
Note, that here we take the worst case bound for the ciphertext
noise, but probabilistic analysis everywhere else. Since 
the key generation is assumed to be honestly performed.

This lower bound on $p_0$ ensure valid decryption in our offline phase.
To obtain a valid and secure set of parameters, we search over
all tuples $(N,p_0,p_1)$ satisfying our various constraints for
a given ``size'' of plaintext modulus $p$; including the upper bound
on $q_1=p_0 \cdot p_1$ obtained from the Albrecht's tool via 
the table above.

The above circuit is needed for obtaining the sharings of the $c$
value, where we need to obtain a fresh encryption of $c$ in order
to be able to obtain the MAC values.
For the obtaining of the MAC values of the $a$ and $b$ values
we need a simpler circuit, and we use the Distributed Decryption
protocol from \cite{KPR}[Appendix A].

\subsection{Zero Knowledge Proof}
The precise ZKPoK we use is from TopGear \cite{TopGear}, which
is an extension of the HighGear prtoocol from \cite{KPR}.
The precise protocol is given in \figref{PZK1} and \figref{PZK2}.
This is an amortized version of the proof from \cite{SPDZ}, in the sense
that the input ciphertexts from all players are simultaneously proved to be
correct.
The proof takes as input a variable $\flag$, which if set to $\Diag$
imposes the restriction that the input plaintexts are ``diagonal'' in
the sense that they encrypt they same value in each slot.
This is needed to encrypt the MAC values at the start of the protcol.
Being diagonal equates to the input plaintext polynomial being the constant
polynomial.



\begin{Boxfig}{Protocol for global proof of knowledge of a set of ciphertexts: Part I}{PZK1}{Protocol $\Pi_{\gZKPoK}$: Commitment, Challenge and Response Phases}
The protocol is parametrized by an integer parameter $U$ and 
a $\flag  \in \left\{ \Diag, \perp \right\}$.

\vspace{3mm}

\noindent
\textsc{Input}:
Each (honest) party $P_i$ enters $U$ plaintexts $\vm_i \in \R_p^U$ (considered
as elements of $\R_{q_1}^U$) and $U$ randomness triples
$R_i \in \R_{q_1}^{U \times 3}$, where each row of $R_i$ 
$(r_i^{(j,1)},r_i^{(j,2)},r_i^{(j,3)})$ is generated from $\RC\left( \sigma^2, 0.5, N \right)$.
For honest $P_i$  we therefore have, for all $j \in [U]$, 
$\norm{\vm_i^{(j)}}_\infty \le \frac{p}{2}$
and 
$\norm{r_i^{(j,k)}}_\infty \le \rho_k$,
where $\rho_1 = \rho_2 =20 $ and $\rho_3=1$.
We write $V=2 \cdot U-1$.

\vspace{3mm}

\noindent{Commitment Phase: $\Comm$}
\begin{enumerate}
\item $P_i$ computes the BGV encryptions by applying the BGV encryption algorithm 
  to the $j$ plaintext/randomness vectors in turn to obtain 
  $C_i \asn \Enc(\vm_i,R_i; \pk) \in \R_{q_1}^{U \times 2}$.
\item The players broadcast $C_i$.
\item Each $P_i$ samples $V$ pseudo-plaintexts $\vy_i \in \R_{q_1}^V$
  and pseudo-randomness vectors $S_i = (s_i^{(j,k)}) \in \R_{q_1}^{V \times 3}$ such that,
for all $j \in [V]$,
$\norm{\vy_i^{(j)}}_\infty \le 2^{\ZKsecp-1} \cdot p$
and
$\norm{s_i^{(j,k)}}_\infty \le 2^\ZKsecp \cdot \rho_k$.
\item Party $P_i$ computes
  $A_i \asn \Enc(\vy_i,S_i; \pk) \in \R_{q_1}^{V \times 2}$.
\item The players broadcast $A_i$.
\end{enumerate}

\noindent{Challenge Phase: $\Challenge$}
\begin{enumerate}
  \item Parties call agree on a random vector $\ve=(e_i)$
    such that $\ve \in \left\{ \left\{ X^i \right\}_{i=0 \ldots, 2 \cdot N-1} \right\}^U$ if $\flag=\perp$ and $\ve \in \{0,1\}^U$ if $\flag=\Diag$.
\end{enumerate}

\noindent{Response Phase: $\Resp$}
\begin{enumerate}
\item Parties define the $V \times U$ matrix $M_\ve$ where
\[
  M_{\ve}^{(k,l)} = \left\{ \begin{array}{ll}
                 e_{k-l+1} & \textrm{if }1\leq k-l+1 \leq U\\
                 0 & \textrm{otherwise}
                     \end{array}
\right.
\]
\item Each $P_i$ computes $ \vz_i \asn \vy_i + M_\ve \cdot \vm_i$
  and $T_i \asn S_i + M_\ve \cdot R_i$. 
\item Party $P_i$ sets $\resp_i \asn \left( \vz_i, T_i \right)$, and broadcasts $\resp_i$.
\end{enumerate}
\end{Boxfig}


\begin{Boxfig}{Protocol for global proof of knowledge of a set of ciphertexts: Part II}{PZK2}{Protocol $\Pi_{\gZKPoK}$: Verification Phase}

\noindent{Verification Phase: $\Verify$}
\begin{enumerate}
\item Each party $P_i$ computes $D_i \asn \Enc( \vz_i, T_i; \pk)$.
\item The parties compute
$A \asn \sum_{i=1}^n A_i$,
$C \asn \sum_{i=1}^n C_i$,
$D \asn \sum_{i=1}^n D_i$,
$T \asn \sum_{i=1}^n T_i$ and
$\vz \asn \sum_{i=1}^n \vz_i$. 
\item The parties check whether $ D = A + M_\ve \cdot C$, and then whether 
  the following inequalities hold, for $j \in [V]$,
\[
    \norm{\vz^{(j)}}_\infty \le n \cdot 2^\ZKsecp \cdot p,
    \quad
    \norm{T^{(j,k)}}_\infty \le 2 \cdot n \cdot 2^\ZKsecp \cdot \rho_k \mbox{ for } k=1,2,3.
\]
\item If $\flag=\Diag$ then the proof is rejected if 
  $\vz^{(j)}$ is not a constant polynomial (i.e. a ``diagonal'' plaintext element).
\item If all checks pass, the parties output $C$.
\end{enumerate}
\end{Boxfig}



Note the soundness security in the case of $\flag=\Diag$ is only
equal to $2^-U$ as opposed to $(2 \cdot \phi(m))^U$.
However, by repeating the ZKPoK a number of times we obtain
the desired soundness security. As this is only done for
the ciphertexts encrypting the MAC keys this is not a problem in
practice.

