\mainsection{FHE Security}\label{sec:fhe}
In this chapter we detail how FHE parameters are selected.
The first part deals with how the basic ``sizes'' are derived,
and the second part gives the mathematical justification (i.e.
noise analysis) for the equations used in the code for setting
up parameters.


\subsection{Main Security Parameters}
Our Ring-LWE samples are (essentially) always from an
approximate Gaussian distribution with standard deviation 3.16
and from a ring with a two-power degree of $N$.
This is not strictly true as some noise samples come
from small Hamming Weight distributions, and some come
from distributions over $\{-1,0,1\}$. 
But the above are the main parameters, and have been used in prior works to
estimate Ring-LWE security in SHE settings.

Given these settings we need to determine the maximum 
(ciphertext) ring modulus $q$ that still gives us security.
For this we use Martin Albrecht's estimator which can
be found at
\begin{center}
\url{https://bitbucket.org/malb/lwe-estimator}
\end{center}
For various values of $n$ and (symmetric equivalent)
security levels \verb+sec+ we then find the minimum secure
value of $q$.
This is done by running the following \verb+sage+ code
\begin{verbatim}
load("estimator.py")
n = 1024
sec = 128
for i in range(10, 500, 5):
   q= 2^i
   costs = estimate_lwe(n, 3.16*sqrt(2*pi)/q, q, reduction_cost_model=BKZ.sieve,  \
                        skip=["arora-gb", "bkw", "dec", "mitm"])
   if any([cost.values()[0]<2^sec for cost in costs.values()]):
       break
print i-5
\end{verbatim}
This will print a bunch of rubbish and then the number $29$. 
This means any $q < 2^{29}$ will be ``secure'' by the above definition of secure.

We did this in Feb 2018 and obtained the following table of values, giving maximum
values of $q$ in the form of $2^x$ for the value $x$ from the following table.
\begin{center}
\begin{tabular}{|c|c|c|c|}
\hline
$N$   & \verb+sec+=80 & \verb+sec+=128 & \verb+sec+=256 \\
\hline
1024  &  44 &    29        & 16 \\
2048  &  86 &    56        & 31 \\
4096  & 171 &   111        & 60 \\
8192  & 344 &   220        & 120 \\
16384 & 690 &   440        & 239 \\
32768 & 998 &   883        & 478 \\
\hline
\end{tabular}
\end{center}
Any updates to this table needs to be duplicated in 
the file \verb+FHE_Params.cpp+ in the code base.

\subsection{Distributions and Norms}
Given an element $a \in R = \Z[X]/(X^{N}+1)$ (represented as a polynomial)
where $X^N+1$ is the $m=2 \cdot N$-th cyclotomic polynomial
(recall $N$ is always a power of two).
We define $\norm{a}_p$ to be the standard $p$-norm
of the coefficient vector (usually for $p=1, 2$ or $\infty$).
We also define $\norm{a}_p^\can$ to be the $p$-norm 
of the same element when mapped into the canonical embedding
i.e.
\[  \norm{a}_p^\can = \norm{\kappa(a)}_p \]
where $\kappa(a): R \longrightarrow \C^{\phi(m)}$ is the
canonical embedding.
The key three relationships are that 
\[ \norm{a}_\infty \le c_m \cdot \norm{a}_\infty^\can \quad
   \mbox{  and  } \quad
   \norm{a}_\infty^\can \le \norm{a}_1  \quad
   \mbox{  and  } \quad
   \norm{a}_1 \le \phi(m) \cdot \norm{a}_\infty.
\]
for some constant $c_m$ depending on $m$.
Since in our protocol we select $m$ to be a power of two then we have $c_m=1$.
In particular (which will be important later in measuring the
noise of potentially dishonestly generated ciphertexts) we have
\[ \norm{a}_{\infty}^{\can} \le \phi(m) \cdot \norm{a}_{\infty}. \]

We also define the \emph{canonical embedding norm reduced modulo~$q$} 
of an element $a\in R$ as the smallest canonical embedding norm of any
$a'$ which is congruent to $a$ modulo~$q$. We denote it as
\[
|a|_q^{can} = \min
  \{~\norm{a'}_\infty^\can :~ a'\in R,~ a' \equiv a \pmod{q}~\}.
\]
We sometimes also denote the polynomial where the minimum is obtained
by $[a]_q^{\can}$, and call it the {\em canonical reduction} of
$a$ modulo $q$. 

Following \cite{GHS12c}[Appendix A.5] we examine the variances
of the different distributions utilized in our protocol.
\begin{itemize}
\item $\HWT(h,N)$: This generates a vector of length $N$
      with elements chosen at random from $\{-1,0,1\}$ subject to
      the condition that the number of non-zero elements is equal to $h$.
\item $\ZO(0.5,N)$:  This generates a vector of length $N$
      with elements chosen from $\{-1,0$, $1\}$ such that the
      probability of coefficient is $p_{-1}=1/4$, $p_0=1/2$
      and $p_1=1/4$.
\item $\dN(\sigma^2,N)$: This generates a vector of
      length $N$ with elements chosen according to an approximation to
      the discrete Gaussian distribution with variance $\sigma^2$.
\item $\RC(0.5,\sigma^2,N)$: This generates a triple of
      elements $(v,e_0,e_1)$ where $v$ is sampled from $\ZO_s(0.5,N)$
      and $e_0$ and $e_1$ are sampled from $\dN_s(\sigma^2,N)$.
\item $\calU(q,N)$: This generates a vector of length $N$ 
      with elements generated uniformly modulo $q$.
\end{itemize}
Let $\zeta_m$ denote any complex primitive $m$-th root of unity.
Sampling $a \in R$ from $\HWT(h,\phi(m))$ and looking at $a(\zeta_m)$
produces a random variable with variance $h$, when sampled
from $\ZO(0.5,\phi(m))$ we obtain variance $\phi(m)/2$,
when sampled from $\dN(\sigma^2,\phi(m))$ we obtain variance
$\sigma^2 \cdot \phi(m)$ and when sampled from $\calU(q,\phi(m))$
we obtain variance $q^2 \cdot \phi(m)/12$.
By the law of large numbers we can use $6 \cdot \sqrt{V}$,
where $V$ is the above variance, as a high probability bound
on the size of $a(\zeta_m)$, and this provides a bound on the
canonical embedding norm of $a$.

If we take a product of two, three, or four such
elements with variances $V_1, V_2, \ldots, V_4$ we use
$16 \cdot \sqrt{V_1 \cdot V_2}$,
$9.6 \cdot \sqrt{V_1 \cdot V_2 \cdot V_3}$ and
$7.3 \cdot \sqrt{V_1 \cdot V_2 \cdot V_3 \cdot V_4}$
as the resulting bounds since 
\[  \mathsf{erfc}(4)^2  \approx \mathsf{erfc}(3.1)^3 
    \approx  \mathsf{erfc}(2.7)^4 \approx  2^{-50}. \]
In our implementation we approximate $\dN(\sigma^2,n)$ using
the binomial method from the NewHope paper, with a standard
deviation of $3.16 = \sqrt{10}$. In particular this means
any vector sampled from $\dN(\sigma^2,n)$ will have
infinity norm bounded by $20$.


\subsection{The FHE Scheme and Noise Analysis}
Note that whilst the scheme is ``standard'' some of the
analysis is slightly different from that presented in the \cite{SPDZ2} paper 
and that presented in the \cite{GHS12c} paper.
We use the notation of the \cite{SPDZ2} paper in this section,
and assume the reader is familiar with prior work.
The key differences are:
\begin{itemize}
\item Unlike in SPDZ-2 we assume a perfect distributed key generation method
amongst the $n$ players.
\item Unlike in SPDZ-2 we are going to use an actively secure ZKPoK (see later),
which will make the actual noise analysis much more complex.
\end{itemize}


\subsubsection{Key Generation:}
The main distinction between the assumptions here and those used in 
the \cite{SPDZ2} paper, is that in the latter a specific distributed 
key generation protocol for the underlying threshold FHE keys was
assumed. In SCALE we assume a `magic black box' which distributes
these keys, which we leave to the application developer to create.
The secret key $\sk$ is selected from a distribution with
Hamming weight $h$, i.e. $\HWT(h,\phi(m))$, 
and then it is distributed amongst the $n$ parties by simply producing a random 
linear combination, and assigning each party one of the sums.
And the switching key data is produced in the standard way, i.e.
in a non-distributed trusted manner.
We assume a two-leveled scheme with moduli $p_0$ and $p_1$ with $q_1=p_0 \cdot p_1$
and $q_0=p_0$.
We require 
\begin{align*}
   p_1 & \equiv 1 \pmod{p}, \\
   p_0 - 1 & \equiv p_1-1 \equiv p-1 \equiv 0 \pmod{\phi(m)}.
\end{align*}
In particular the public key is of the form $(a,b)$ where
\[ a \asn \calU(q,\phi(m)) \quad \mbox{ and } \quad b = a \cdot \sk + p \cdot \epsilon \]
where $\epsilon \asn \dN(\sigma^2,\phi(m))$ 
and the switching key data $(a_{\sk,\sk^2},b_{\sk,\sk^2})$ is of the form
\[ a_{\sk,\sk^2} \asn \calU(q,\phi(m)) \quad \mbox{ and } \quad 
   b_{\sk,\sk^2} =   a_{\sk,\sk^2} \cdot \sk + p \cdot e_{\sk,\sk^2} - p_1 \cdot \sk^2 \]
where $e_{\sk,\sk^2} \asn \dN(\sigma^2,\phi(m))$.
We take $\sigma=3.16$ as described above in what follows.

\subsubsection{Encryption:}
To encrypt an element $m\in R$, we choose $v, e_0, e_1 \asn \RC(0.5,\sigma^2,n)$, i.e.
\[
v \asn \ZO(0.5, \phi(m)) ~~~\mbox{and}~~ e_0,e_1 \asn \dN(\sigma^2,\phi(m)) 
\]
Then we set $c_0 = b \cdot v + p \cdot e_0+m$,~ $c_1=a\cdot v+p\cdot
e_1$, and set the initial ciphertext as $\ct'=(c_0,c_1)$.
We calculate a bound (with high probability) on the output noise of
an honestly generated ciphertext to be
\begin{align*}
  \norm{c_0 - \sk \cdot c_1}_\infty^\can 
     &= \norm{((a \cdot \sk+ p \cdot \epsilon) \cdot v +p \cdot e_0 +m
                - (a \cdot v+ p \cdot e_1) \cdot \sk }_\infty^\can \\
     &= \norm{m + p \cdot (\epsilon \cdot v +e_0 - e_1 \cdot \sk)}_\infty^\can \\
     &\le \norm{m}_\infty^\can 
         +p \cdot \left( \norm{\epsilon\cdot v}_\infty^\can
                       + \norm{e_0}_\infty^\can
                       + \norm{e_1 \cdot \sk}_\infty^\can
                  \right) \\
     &\le \phi(m) \cdot p/2
	  + p \cdot \sigma \cdot 
             \left( 16 \cdot \phi(m) / \sqrt{2}
			  + 6 \cdot \sqrt{\phi(m)}
			  +16 \cdot \sqrt{h \cdot \phi(m)}
		    \right) = B_\clean.
\end{align*}
Note this is a probablistic bound and not an absolute bound.

\vspace{5mm}

\noindent
However, below we will only be able to gaurantee $m, v, e_0$ and $e_1$
are selected subject to
\[ \norm{v}_\infty \le 2^{3 \cdot \secp/2+1} \cdot n \quad 
    \mbox{  and  } \quad
   \norm{e_0}_\infty, \norm{e_1}_\infty \le 20 \cdot 2^{3 \cdot \secp/2+1} \cdot n \quad
    \mbox{  and  } \quad
   \norm{m}_\infty \le 2^{3 \cdot \secp/2+1} \cdot n \cdot p/2,
\]
where $\secp$ is our statistical security parameter.
In this situation we obtain the bound, using the inequality above
between the infinity norm in the polynomial embedding
and the infinity norm in the canonical embeddig,
\begin{align*}
 \norm{c_0 - \sk \cdot c_1}_\infty^\can 
   & \le  \norm{m}_\infty^\can 
         +p \cdot \left( \norm{\epsilon\cdot v}_\infty^\can
                       + \norm{e_0}_\infty^\can
                       + \norm{e_1 \cdot \sk}_\infty^\can
                  \right) \\
   &\le  \norm{m}_\infty^\can 
         +p \cdot \left( \norm{\epsilon}_\infty^\can \cdot \norm{ v}_\infty^\can
                       + \norm{e_0}_\infty^\can
                       + \norm{e_1}_\infty^\can \cdot \norm{ \sk}_\infty^\can
                  \right)  \\
   &\le \phi(m) \cdot 2^{3 \cdot \secp/2+1} \cdot  n  \cdot p \cdot
	\left( 1/2 + 20 \cdot 6 \cdot \sigma \cdot \sqrt{\phi(m)}
		   + 20 
		   + 20 \cdot 6 \cdot \sqrt{h}
        \right) = B_\clean^\dishonest
\end{align*}
Again this is a probabilistic bound (assuming validly
distributed key generation), but assumes the worst case
we the ciphertext bounds.

\subsubsection{$\SwitchModulus((c_0,c_1),\lev)$:}
This takes as input a ciphertext modulo $q_\lev$ and outputs a ciphertext mod $q_{\lev-1}$".
The initial ciphertext is at level $q_1=p_0 \cdot p_1$, with $q_0=p_0$.
If the input ciphertext has noise bounded by $\nu$
in the canonical embedding
then the output ciphertext will have noise bounded by $\nu'$ in
the canonical embedding, where
\[ \nu' = \frac{\nu}{p_\lev} + B_\scale. \]
The value $B_\scale$ is an upper bound on the quantity
$\norm{\tau_0+\tau_1 \cdot \sk}_\infty^\can$, where
$\kappa(\tau_i)$ is drawn from a distribution
which is close to a complex Gaussian with variance $\phi(m)\cdot p^2/12$.
We therefore, we can (with high probability) take the upper
bound to be
\begin{align*}
    B_\scale 
	&= 6 \cdot p \cdot \sqrt{ \phi(m)/12}
	+ 16 \cdot p \cdot \sqrt{ \phi(m) \cdot h/12}, \\
        &= p \cdot \sqrt{3 \cdot \phi(m)} \cdot 
		\left( 1 + 8 \cdot \sqrt{h}/3 \right).
\end{align*}
This is again a probabilistic analysis, assuming validly generated
public keys.

\subsubsection{$\Dec_{\sk}(\ct)$:}
As explained in \cite{SPDZ2,GHS12c} this procedure works when the noise 
$\nu$ (in the canonical embedding) associated with a ciphertext satisfies $c_m \cdot \nu  < q_{\ell}/2$.
However, as we always take power of two cyclotomics we have $c_m=1$

\subsubsection{$\DistDec_{\{\sk_i\}}(\ct)$:}
There are two possible distributed decryption protocols.
The first one is from \cite{SPDZ} is for when we want to obtain a
resharing of an encrypted value, along with a fresh ciphertext.
The second version is from \cite{KPR}, where we do not need to
obtain a fresh encryption of the value being reshared.


\begin{Boxfig}{The sub-protocol for additively secret sharing a plaintext $\vm \in (\F_{p^k})^s$ on input a ciphertext $e_\vm=\Enc_\pk(\vm)$.}{reshare}{$\mathsf{Reshare-1}$}
Input is $e_\vm$, where $e_\vm=\Enc_\pk(\vm)$ is a public ciphertext.

Output is a share $\vm_i$ of $\vm$ to each player $P_i$; and a ciphertext $e'_\vm$. 

The idea is that $e_\vm$ could be a product of two ciphertexts, which
$\mathsf{Reshare}$ converts to a ``fresh'' ciphertext $e'_\vm$. Since $\mathsf{Reshare}$ uses distributed decryption (that may return an incorrect result), it is not guaranteed that $e_\vm$ and $e'_\vm$ contain the same value, but it
{\em is} guaranteed that $\sum_i \vm_i$ is the value contained in $e'_\vm$.
\begin{enumerate}
\item Each player $P_i$ samples a uniform $\vf_i\in (\F_p)^N$. Define $\vf:=\sum_{i=1}^{n}\vf_i$.
\item Each player $P_i$ computes and broadcasts $e_{\vf_i}\asn\Enc_\pk(\vf_i)$.\label{reshare:enc}
\item Each player $P_i$ runs the proof below as a prover on $e_{\vf_i}$. The protocol aborts if any proof fails
and if successful each player obtains $e_\vf\asn e_{\vf_1}\boxplus\dots\boxplus e_{\vf_n}$.
\item The players compute $e_{\vm+\vf}\asn e_\vm\boxplus e_\vf$.
\item The players decrypt $e_{\vm+\vf}$ as follows:
\begin{enumerate}
\item Player one computes $\vv_1 = e_{\vm+\vf}^{(0)}-\sk_1 \cdot e_{\vm+\vf}^{(1)}$ and player $i \ne 1$ computes.
$\vv_i = -\sk_i \cdot e_{\vm+\vf}^{(1)}$.
\item All players compute $\vt_i = \vv_i + p \cdot \vr_i$ for some random element
with infinity norm given by $2^\secp \cdot B/p$.
\item The parties broadcast $\vt_i$.
\item The parties compute $\vm+\vf = \sum \vt_i \pmod{p}$.
\end{enumerate}
\item $P_1$ sets $\vm_1\asn \vm+\vf-\vf_1$, and
each player $P_i$ ($i\neq 1$) sets $\vm_i\asn -\vf_i$.
%\item Each player $P_i$ now holds a share $m_i$ of the secret value $m$.
\item All players set $e'_\vm \asn \Enc_{pk}(\vm+\vf)\boxminus e_{\vf_1} \boxminus \dots \boxminus e_{\vf_n}$,
where a default value for the randomness is used when computing $\Enc_{pk}(\vm+\vf)$.
\end{enumerate}
\end{Boxfig}


\paragraph{Reshare Version 1:}
This is described in Figure \ref{reshare}.
The value $B$ in the protocol is an upper bound on the noise in the canonical embedding
$\nu$ associated with a ciphertext we will decrypt in our protocols.
To ensure valid distributed decryption we require
\[ 2 \cdot (1+n \cdot 2^\secp) \cdot B < q_{\ell}. \]

Given a value of $B$, we therefore will obtain a lower bound
on $p_0$ by the above inequality.
The addition of a random term with infinity norm bounded by
$2^\secp \cdot B/p$ in the distributed decryption procedure
ensures that the individual {\em coefficients} of the sum
$\vt_1+\cdots+\vt_n$ are statistically indistinguishable from
random, with probability $2^{-\secp}$.
This does not imply that the adversary has this probability of
distinguishing the simulated execution in \cite{SPDZ} from the
real execution; since each run consists of the exchange of 
$\phi(m)$ coefficients, and the protocol is executed many times
over the execution of the whole protocol.
We however feel that setting concentrating solely on the
statistical indistinguishability of the coefficients is valid 
in a practical context.

\paragraph{Reshare Version 2:}
This is given in Figure \ref{distdec}.
This protocol is simpler as it does not require the resharing to a 
new ciphertext.
So in particular players do not need to provide encryptions of
their $\vf_i$ values, and hence there is no need
to execute the ZKPoK needed in the previous protocol.

\begin{Boxfig}{Distributed decryption to secret
    sharing}{distdec}{$\mathsf{Reshare-2}$}
Furthermore, let $B$ denote a bound on the noise, that is $\norm{c_0 - s \cdot c_1}_\infty$.
\begin{enumerate}
\item Party $i$ samples $\vf_i \asn [0,B\cdot 2^\secp]^N$.
\item Party $1$ computes
  $\vv_i := (c_0 - s_1 \cdot c_1) - \vf_1 \bmod q$, and every other
  party $i$ computes $\vv_i :=  -s_i \cdot c_1- \vf_i$.
\item Party $i$ broadcasts $\vv_i$.
\item Party 1 outputs
  $\vm_1 := (\sum_{i=1}^n \vv_i \bmod q + \vf_1) \bmod p$, and
  every other party $i$ outputs $\vm_i := \vf_i \bmod p$.
\end{enumerate}
\end{Boxfig}





\subsubsection{$\SwitchKey(d_0,d_1,d_2)$:}
In order to estimate the size of the output noise term 
in the canonical embedding we need first to estimate the size of the term
(again probabilistically assuming validly generated public
keys)
\[ \norm{p \cdot d_2 \cdot \epsilon_{\sk,\sk^2}}_\infty^\can. \]
\begin{align*}
 \norm{ p \cdot d_2 \cdot e_{\sk,\sk^2}}_\infty^\can /q_0
     &\le p \cdot \sqrt{\frac{\phi(m)}{12}} \cdot
          \left[ \sigma \cdot \left(
                        7.3 \cdot \sqrt{h \cdot \phi(m)^2/2}  %  7.3 e vj si 
	              + 9.6 \cdot \sqrt{h \cdot \phi(m)}              % 9.6 e0 si
			\right. \right. \\
                &~~~~~~~~~~~~~~~~~~~~~~~~~~~~~~~~~~~~~~~~~~~~~~~~~~~~~~~ \left. \left.
                      + 7.3 \cdot h \cdot \sqrt{\phi(m)}      % 7.3 s si e1
			   \right) \right.  \\
		&~~~~~~~~~~~~~~~~~~~~~~~~~~~~~~\left. + \left(
			9.6 \cdot \sigma \cdot \sqrt{\phi(m)^2/2}          % 9.6 e vi'
                      + 16  \cdot \sigma \cdot \sqrt{\phi(m)}                      % 16 e0'
			\right. \right. \\
                & ~~~~~~~~~~~~~~~~~~~~~~~~~~~~~~~~~~~~~~~~~~~~~~~~~~~~~~~ \left. \left.
                      + 7.6 \cdot \sigma \cdot \sqrt{\phi(m) \cdot h}      % 7.6 e1' s
			  \right)  \right] \\
     &\le  p \cdot \phi(m) \cdot \sigma  \cdot
          \left[              1.49 \cdot \sqrt{h \cdot \phi(m)} + 2.11 \cdot h + 5.54 \cdot \sqrt{h} 
			   +  1.96 \cdot \sqrt{\phi(m)}  
                      + 4.62 
	  \right] \\
     & = B_\KS.
\end{align*}
Then if the input to $\SwitchKey$ has noise bounded by $\nu$ then the output 
noise value in the canonical embedding will be bounded by
\[ \nu+\frac{B_\KS \cdot q_0}{p_1} + B_\scale. \]


\subsubsection{$\Mult(\ct,\ct')$:}
Combining the all the above, if we take two ciphertexts of level one
with input noise in the canonical embedding bounded by $\nu$ and $\nu'$, the output noise level 
from multiplication will be bounded by
\[ \nu'' =      \left( \frac{\nu}{p_1} + B_\scale \right)
	  \cdot 
                \left( \frac{\nu'}{p_1} + B_\scale \right)
		+ \frac{B_\KS}{p_1}+B_\scale.
\]

\subsubsection{Application to the Offline Phase:}
In all of our protocols the most complext circuit we will be evaluating 
is the following one: 
We first add $n$ ciphertexts together and perform a multiplication, giving a 
ciphertext with respect to modulus $p_0$ with noise in the canonical
embedding bounded by
\[
  U_1 = \left( \frac{n \cdot B_\clean^\dishonest}{p_1}+B_\scale \right)^2
		+ \frac{B_{\KS} \cdot p_0}{p_1} + B_\scale.
\]
We then add on another $n$ ciphertexts, which are added at 
level one and then reduced to level zero.
We therefore obtain a final upper bound on the noise 
in the canonical embedding for our adversarially generated ciphertexts of
\[ U_2 = U_1 +  \frac{n \cdot B_\clean^\dishonest}{p_1} + B_\scale. \]
To ensure valid (distributed) decryption, we require
\[ 2 \cdot U_2 \cdot (1+n \cdot 2^\secp) < p_0, \]
i.e. we take $B=U_2$ in our distributed decryption protocol.
Note, that here we take the worst case bound for the ciphertext
noise, but probabilistic analysis everywhere else. Since 
the key generation is assumed to be honestly performed.

This lower bound on $p_0$ ensure valid decryption in our offline phase.
To obtain a valid and secure set of parameters, we search over
all tuples $(N,p_0,p_1)$ satisfying our various constraints for
a given ``size'' of plaintext modulus $p$; including the upper bound
on $q_1=p_0 \cdot p_1$ obtained from the Albrecht's tool via 
the table above.

The above circuit is needed for obtaining the sharings of the $c$
value, where we need to obtain a fresh encryption of $c$ in order
to be able to obtain the MAC values.
For the obtaining of the MAC values of the $a$ and $b$ values
we need a simpler circuit, and we use the Distributed Decryption
protocol from \cite{KPR}[Appendix A].

\subsection{Zero Knowledge Proof}
Here we give the ZK-PoK from Uverdrive \cite{KPR}, with some modifications
to make it clearer what is going on, and to make 
consistent with the notation above.
This is an amortized version of the proof from \cite{SPDZ}, in the sense
that the input ciphertexts from all players are simulataneously proved to be
correct.
The proof takes as input a variable \verb+diag+, which if set to true
imposes the restriction that the input plaintexts are ``diagonal'' in
the sense that they encrypt they same value in each slot.
This is needed to encrypt the MAC values at the start of the protcol.
Being diagonal equates to the input plaintext polynomial being the constant
polynomial.

\begin{Boxfig}{Protocol for global proof of knowledge of a ciphertext}{Pgpok}{$\Pgpok$}
We wish to prove that a set of $\secp$ ciphertexts $\vct\ui=\ct\ui_j$ for $j=1,\ldots, \sec$
input by party $i$ (for all $i$) are generated such that
\[ \norm{v\ui_j}_\infty \le 2^{3 \cdot \secp/2+1} \cdot n \quad 
    \mbox{  and  } \quad
   \norm{e\ui_{0,j}}_\infty, \norm{e\ui_{1,j}}_\infty \le 20 \cdot 2^{3 \cdot \secp/2+1} \cdot n \quad
    \mbox{  and  } \quad
   \norm{m\ui_j}_\infty \le 2^{3 \cdot \secp/2+1} \cdot n \cdot p/2,
\]
We will write $\vr\ui_j=(v\ui_j, e\ui_{0,j}, e\ui_{1,j})$
and $\vx\ui_j=(m\ui_1,\ldots,m\ui_\secp)$.
Thus $\vx\ui \in R^\secp$, $\vr\ui \in R^{\secp \cdot 3}$ and $\ct\ui\in R^{\secp \cdot 2}$.
\begin{enumerate}
\item Set $V= 2 \cdot \secp-1$.
\item \label{stage1} 
Each party $P_i$ samples a new set of ``plaintexts'' $\vy\ui \in R^V$ and ``randomness vectors''
$\vs\ui \in R^{V \times 3} $ such that
\[ \norm{y_j}_\infty \le 2^\secp \cdot p/2 \quad \mbox{  and  } \quad
   \norm{s_{j,1}}_\infty \le 2^\secp \quad \mbox{  and   } \quad
   \norm{s_{j,2}}_\infty, \norm{s_{j,3}}_\infty \le 2^\secp \cdot 20.
\]
for $j=1,\ldots,V$.
If \verb+diag+ is set to true, then the $y_j$ are selected to be diagonal.
\item Each $P_i$ computes $\va\ui \asn \Enc_\pk(\vy\ui,\vs\ui)$ and broadcasts the $\va\ui$.
\item The parties jointly sample a vector $\ve \in \{0,1\}^\secp$.
\item Define $M_\ve \in \{0,1\}^{V \times \secp}$ to be the matrix associated with the challenge $\ve$ 
such that $M_{kl} = e_{k-l+1}$ for $1 \leq k-l+1 \leq \secp$ and $0$ in all other entries.
\item Each party $P_i$ computes $\vz\uti = \vy\uti + M_\ve \cdot \vx\uti \in R^{V}$ 
and $T\ui=\vs\ui + M_\ve \cdot \vr\ui \in R^{V \times 3}$ and broadcasts $(\vz\ui,T\ui)$.
\item \label{stage20} Each party $P_i$ computes $\vd\ui = \Enc_\pk(\vz\ui, T\ui)$ 
and then stores the sum $\vd = \sum_{i=1}^n \vd\ui$.
\item \label{stage21} 
The parties compute $\vct =\sum_i \vct\ui$,~ $\va = \sum_i \va\ui$,~ $\vz = \sum_i \vz\ui$ and $T = \sum_i T\ui$ and 
conduct the checks (allowing the norms to be $2n$ times bigger to accommodate the summations):
\[\transpose{\vd} = \transpose{\va} + (M_\ve \cdot \vct), \quad
  \norm{\vz}_\infty \leq 2 \cdot n \cdot 2^\secp \cdot p/2, \quad
  \norm{t_{i,1}}_\infty \leq 2 \cdot n \cdot 2^{\secp}, \quad
  \norm{t_{i,2}}_\infty, \norm{t_{i,3}}_\infty \leq 2 \cdot 20 \cdot n \cdot 2^{\secp}, \quad
\]
where $t_{i,j}$ is the $(i,j)$-th element of $T$.
\item If \verb+diag+ is true and the $\vz\ui$ are not diagonal plaintexts then the
proof is rejected.
\item If the check passes, the parties output $\sum_{i=1}^{n} \vct\ui$.
  \end{enumerate}
\end{Boxfig}
In the following we will prove that our protocol achieves the natural
extension of the $\Sigma$-protocol properties in the multi-party
setting.
We ignore the \verb+diag+ flag, as the modifications in the case
this is set to true are trivial.

\paragraph{Correctness.}
The equality in step \ref{stage21} follows trivially from the linearity of the encryption.
It remains to check the probability that an honest prover will fail the bounds check on $\norm{\vz}_\infty$ 
and $\norm{t_{i,j}}_\infty$.

Remember that the honestly generated $\ct\ui$ are ciphertexts
generated according to the true distribution (i.e. without the slack).

The bound check for the plaintext component will succeed if the 
infinity norm of $\sum_{i=1}^n (\vy^{(i)} + \sum_{k=1}^{\secp}(M_{e_{jk}} \cdot
\vx^{(i)}))$ is at most $2 \cdot n \cdot 2^\secp \cdot p/2$. 
This is always true because $\vy^{(i)}$ is sampled such that 
$\norm{\vy^{(i)}}_\infty \leq 2^{\secp} \cdot p/2$ and 
$\norm{M_\ve \cdot \vx^{(i)}}_\infty \le \secp \cdot p/2 \le 2^\secp \cdot p/2$. 
A similar argument holds regarding $\rho$.

\paragraph{Special soundness.} To prove this property one must be able to extract the witness given responses from 
two different challenges. 
In this case consider the transcripts $(\vx, \va, \ve, (\vz, T))$ and $(\vx, \va, \ve', (\vz', T'))$ where $\ve \neq \ve'$. 
Recall that each party has a different secret $\vx\ui$. 
Because both challenges have passed the bound checks during the protocol, we get that:
\begin{align*}
    (M_\ve - M_{\ve'}) \cdot \transpose{\vct} = \transpose{(\vd - \vd')}
\end{align*}

To solve the equation for $\vct$ notice that $M_\ve - M_{\ve'}$ is a matrix with entries in $\{-1,0,1\}$ so we 
must solve a linear system where $\vct = \Enc_\pk(\vx_k, \vr_k)$ for $k = 1,\dots,\secp$. 
This can be done in two steps: solve the linear system for the first half: $\vc_1, \dots, \vc_{\secp/2}$ and 
then for the second half: $\vc_{\secp/2+1}, \dots, \vc_{\secp}$. 
For the first step identify a square submatrix of $\secp \times \secp$ entries in $M_\ve - M_{\ve'}$ which 
has a diagonal full of $1$'s or $-1$'s and it is lower triangular. 
This can be done since there is at least one component $j$ such that $e_j \neq e'_j$. 
Recall that the plaintexts $\vz_k, \vz'_k$ have norms less than $2^\secp \cdot p/2$ and 
the randomness used for encrypting them, $\vt_k, \vt'_k$, have norms less than 
$2^\secp$ in the first coordinate and $2^\secp \cdot 20$ in the last two coordinates where $k$ ranges 
through $1, \dots, \secp$.

Solving the linear system from the top row to the middle row via substitution we obtain in the worst case: 
$\norm{\vx_k}_\infty \leq 2^k \cdot n \cdot 2^\secp \cdot p/2$ and the infinity norm of
$\vy_k$ is less than $2^k \cdot n \cdot 2^\secp$ in the first coordinate
and less than $2^k \cdot n \cdot 2^\secp \cdot 20$ in the last two coodinates,
where $k$ ranges through $1, \dots, \secp/2$. 

The second step is similar to the first with the exception that now we have to look for an 
upper triangular matrix of $\secp \times \secp$. 
Then solve the linear system from the last row to the middle row. 
In this way we extract $\vx_k, \vr_k$ which we can only gaurantee satisfy the 
bounds given in Figure \ref{Pgpok}.

\paragraph{Honest verifier zero-knowledge.} 
Here we give a simulator $\Sim$ for an honest verifier (each party $P_i$ acts as one at one point during the protocol). 
The simulator's purpose is to create a transcript with the verifier which is indistinguishable from the real 
interaction between the prover and the verifier. 
To achieve this, $\Sim$ samples uniformly $\ve \asn \{0,1\}^{\secp}$ and then creates the transcript accordingly: 
sample $\vz\ui$ and $T\ui$ with respect to the bounds in the final check.
The simulator then fixes $\va\ui = \Enc_\pk(\vz\ui, T\ui) - (M_\ve \cdot \vct\ui)$, where the encryption is applied component-wise. Clearly the produced transcript $(\va\ui, \ve\ui, \vz\ui, T\ui)$ passes the final checks and the statistical distance to the real one is $2^{-\secp}$, which is negligible with respect to $\secp$.

