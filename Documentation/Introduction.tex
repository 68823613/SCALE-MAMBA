

\mainsection{Introduction}\label{sec:introduction}
The SCALE system consists of three main sub-systems:
An offline phase, an online phase and a compiler. 
Unlike the earlier SPDZ system, in SCALE the online
and offline phases are fully integrated. Thus you
can no longer time just the online phase, or just
the offline phase.
The combined online/offline phase we shall refer
to as SCALE, the compiler takes a program written
in our special language MAMBA, and then turns it
into bytecode which can be executed by the SCALE
system.

We provide switches (see below) to obtain 
different behaviours between how the online and offline
phases are integrated together, which can allow for
some form of timing approximation.
The main reason for this change is to ensure that the
system is ``almost'' secure out of the box, even if
means it is less good for obtaining super-duper numbers
for research papers.

An issue though is that the system takes a while to warm
up the offline queues before the online phase can execute.
This is especially true in the case of using a Full Threshold
secret sharing scheme. Indeed in this case it is likely
that the online phase will run so-fast that the offline
phase is always trying to catch up. In addition, in this
situation the offline phase needs to do a number of
high-cost operations before it can even start. Thus
using the system to run very small programs is going to
be inefficient, although the execution time you get
is indicative of the total run time you should be getting
in a real system, it is just not going to be very
impressive.

In order to enable efficient operation in the case where the
offline phase is expensive (e.g. for Full Threshold secret sharing)
we provide an {\em experimental} mechanism to enable the SCALE system to run
as a seperate process (generating offline data), and then
a MAMBA program can be compiled in a just-in-time manner
and can then be dispated to the SCALE system for
execution. Our methodology for this is not perfect, but has
been driven by a real use case of the system.
See Section \ref{sec:restart} for more details.

But note SCALE/MAMBA is {\em an experimental research system} 
there has been no effort to ensure that the system meets rigourous production
quality code control. 
In addition this also means it comes with limited support.
If you make changes to any files/components you are on your
own.
If you have a question about the system we will endeavour to
answer it.

\vspace{5mm}

\noindent
{\bf Warnings:}
\begin{itemize}
\item
The Overdrive system \cite{KPR} for the offline phase
for full-threshold access structures requires a distributed key 
generation phase for the underlying homomorphic encryption scheme. 
The SPDZ-2 system and paper does describe such a protocol, but it is only
covertly secure. 
We do not provide the protocol to do this, instead the
Setup program in this instance will generate a suitable
key and distribute it to the different players.
In any real system this entire setup phase will need
investigating, with perhaps using HSMs to construct and deploy
keys if no actively secure key generation protocol is available.
\item
There is a security hole in how we have implemented things.
As the
\begin{center}
  \verb+offline->sacrifice->online+ 
\end{center}
pipeline is run continously in separate threads, each with their own channels etc., we 
may {\bf use} a data item in the online phase {\bf before} the checking
of a data item has {\bf fully completed} (i.e. before in an associated
sacrifice etc. has been MAC-checked, for full-threshold,
or hash-checked, for other LSSS schemes). 
In a real system you will want to address this by having some
other list of stuff (between sacrifice and online), which
ensures that all checks are complete before online is
allowed to use any data; or by ensuring MAC/hash-checking is
done before the sacrifice queues are passed to the online phase.
In our current system if something bad is introduced by an adversary
then the system {\bf will} halt.
But, before doing so there is a {\em small} chance that some data will 
have leaked from the computation if the adversary can schedule the 
error at exactly the right point (depending on what is happening in
other threads).
\end{itemize}


\subsection{Architecture}
The basic internal runtime architecture is as follows:
\begin{itemize}
\item Each MAMBA program (\verb+.mpc+ file) will initiate a number of threads
to perform the online stage.
The number of ``online threads'' needed is worked out by the compiler. You
can programmatically start and stop threads using the python-like
language (see later).
Using multiple threads enables you to get high throughput. Almost all
of our experimental results are produced using multiple threads.
\item Each online is associated with another four ``feeder'' threads.
One produces multiplication triples, one produces square pairs
and one produces shared bits.
The fourth thread performs the sacrificing step, as well as
the preprocessing data for player IO.
The chain of events is that the multiplication thread produces
an unchecked triple. This triple is added to a triple-list (which is
done in batches for efficiency).
At some point the sacrifice thread decides to take some
data off the triple-list and perform the triple checking via
sacrificing.
Once a triple passes sacrificing it is passed onto another
list, called the sacrificed-list, for consumption by the online phase.
\item By having the production threads aligned with an online
thread we can avoid complex machinary to match produces with
consumers. This however may (more like will) result in the 
over-production of offline data for small applications.
\item In the case of Full Threshold we have another set of global
threads (called the FHE Factory threads, or the FHE Industry) 
which produces level one random FHE ciphertexts which have passed 
the ZKPoK from the high gear variant of Overdrive \cite{KPR}.
This is done in a small bunch of global threads as the ZKPoK would
blow up memory if done for each thread that would consume 
data from the FHE thread.
Any thread (offline production thread) can request a new
ciphertext/plaintext pair from the FHE factory threads.
\end{itemize}


