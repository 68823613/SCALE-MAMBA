\mainsection{Programmatic Restarting}
\label{sec:restart}
This feature is totally {\em experimental} and may change in future
releases.

Because, unlike in SPDZ, the offline and online phases are totally
integrated, it can takes a long time for the queues of pre-processed
triples to be full. This means that quickly starting an instance to
run a short program can be very time-consuming. For reactive systems
in which the precise program which is going to be run is not known
until just before execution this can be a problem.
We therefore provide a mechanism, which we call \verb+RESTART+, to
enable a program to {\em programatically} restart the online runtime, whilst 
still maintaining the current offline data.
In the bytecode this is accessed by the op-code \verb+RESTART+
and in MAMBA it is accessed by the function \verb+restart()+.

The basic model is as follows.
You start an instance of the SCALE engine with some (possibly
dummy) program. This program instance will define the total number
of online threads you will ever need; this is because we do not
want a restart operation to have to spawn new offline threads.
You then ensure that the last operation performed by this
program is a \verb+RESTART+/\verb+restart()+.
At this point the runtime waits for a trigger to be obtained
from {\em each} player on the \verb+IO_FUNCTIONALITY+ (see
Section \ref{sec:IO}).
This trigger can be anything, but in our default IO class it
is simply each player needing to enter a value on the standard
input.

At this point the schedule file is reloaded from the 
{\em same directory} as the original program instance.
This also reloads the tapes etc.
The reason for this triggering is that the system which is
calling SCALE {\em may} want to replace the underlying
schedule file and tapes; thus enabling the execution of
a new program entirely.

To see this in operation we provide the following simple
example, in \verb+Programs/restart_1/restart.mpc+
and \verb+Programs/restart_2/restart.mpc+.
To see these in operation execute the following commands
from the main directory
\begin{verbatim}
            \cp Programs/restart_1/restart.mpc Programs/restart/
            ./compile.py Programs/restart
\end{verbatim}
Now run the resulting program as normal, i.e. using \verb+Programs/restart+
as the program.
When this program has finished it asks (assuming you are
using the default IO class) for all players to enter
a number. {\em Do not do this yet!}
First get the next program ready to run by executing
\begin{verbatim}
        \cp Programs/restart_2/restart.mpc Programs/restart/
        ./compile.py Programs/restart
\end{verbatim}
Now enter a number on each players console, and 
the second program should now run.
Since it also ends with a \verb+restart()+ command, you can
continue in this way forever.
