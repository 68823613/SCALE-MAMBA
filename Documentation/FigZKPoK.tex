

\begin{Boxfig}{Protocol for global proof of knowledge of a set of ciphertexts: Part I}{PZK1}{Protocol $\Pi_{\gZKPoK}$: Commitment, Challenge and Response Phases}
The protocol is parametrized by an integer parameter $U$ and 
a $\flag  \in \left\{ \Diag, \perp \right\}$.

\vspace{3mm}

\noindent
\textsc{Input}:
Each (honest) party $P_i$ enters $U$ plaintexts $\vm_i \in \R_p^U$ (considered
as elements of $\R_{q_1}^U$) and $U$ randomness triples
$R_i \in \R_{q_1}^{U \times 3}$, where each row of $R_i$ 
$(r_i^{(j,1)},r_i^{(j,2)},r_i^{(j,3)})$ is generated from $\RC\left( \sigma^2, 0.5, N \right)$.
For honest $P_i$  we therefore have, for all $j \in [U]$, 
$\norm{\vm_i^{(j)}}_\infty \le \frac{p}{2}$
and 
$\norm{r_i^{(j,k)}}_\infty \le \rho_k$,
where $\rho_1 = \rho_2 =20 $ and $\rho_3=1$.
We write $V=2 \cdot U-1$.

\vspace{3mm}

\noindent{Commitment Phase: $\Comm$}
\begin{enumerate}
\item $P_i$ computes the BGV encryptions by applying the BGV encryption algorithm 
  to the $j$ plaintext/randomness vectors in turn to obtain 
  $C_i \asn \Enc(\vm_i,R_i; \pk) \in \R_{q_1}^{U \times 2}$.
\item The players broadcast $C_i$.
\item Each $P_i$ samples $V$ pseudo-plaintexts $\vy_i \in \R_{q_1}^V$
  and pseudo-randomness vectors $S_i = (s_i^{(j,k)}) \in \R_{q_1}^{V \times 3}$ such that,
for all $j \in [V]$,
$\norm{\vy_i^{(j)}}_\infty \le 2^{\ZKsecp-1} \cdot p$
and
$\norm{s_i^{(j,k)}}_\infty \le 2^\ZKsecp \cdot \rho_k$.
\item Party $P_i$ computes
  $A_i \asn \Enc(\vy_i,S_i; \pk) \in \R_{q_1}^{V \times 2}$.
\item The players broadcast $A_i$.
\end{enumerate}

\noindent{Challenge Phase: $\Challenge$}
\begin{enumerate}
  \item Parties call agree on a random vector $\ve=(e_i)$
    such that $\ve \in \left\{ \left\{ X^i \right\}_{i=0 \ldots, 2 \cdot N-1} \right\}^U$ if $\flag=\perp$ and $\ve \in \{0,1\}^U$ if $\flag=\Diag$.
\end{enumerate}

\noindent{Response Phase: $\Resp$}
\begin{enumerate}
\item Parties define the $V \times U$ matrix $M_\ve$ where
\[
  M_{\ve}^{(k,l)} = \left\{ \begin{array}{ll}
                 e_{k-l+1} & \textrm{if }1\leq k-l+1 \leq U\\
                 0 & \textrm{otherwise}
                     \end{array}
\right.
\]
\item Each $P_i$ computes $ \vz_i \asn \vy_i + M_\ve \cdot \vm_i$
  and $T_i \asn S_i + M_\ve \cdot R_i$. 
\item Party $P_i$ sets $\resp_i \asn \left( \vz_i, T_i \right)$, and broadcasts $\resp_i$.
\end{enumerate}
\end{Boxfig}


\begin{Boxfig}{Protocol for global proof of knowledge of a set of ciphertexts: Part II}{PZK2}{Protocol $\Pi_{\gZKPoK}$: Verification Phase}

\noindent{Verification Phase: $\Verify$}
\begin{enumerate}
\item Each party $P_i$ computes $D_i \asn \Enc( \vz_i, T_i; \pk)$.
\item The parties compute
$A \asn \sum_{i=1}^n A_i$,
$C \asn \sum_{i=1}^n C_i$,
$D \asn \sum_{i=1}^n D_i$,
$T \asn \sum_{i=1}^n T_i$ and
$\vz \asn \sum_{i=1}^n \vz_i$. 
\item The parties check whether $ D = A + M_\ve \cdot C$, and then whether 
  the following inequalities hold, for $j \in [V]$,
\[
    \norm{\vz^{(j)}}_\infty \le n \cdot 2^\ZKsecp \cdot p,
    \quad
    \norm{T^{(j,k)}}_\infty \le 2 \cdot n \cdot 2^\ZKsecp \cdot \rho_k \mbox{ for } k=1,2,3.
\]
\item If $\flag=\Diag$ then the proof is rejected if 
  $\vz^{(j)}$ is not a constant polynomial (i.e. a ``diagonal'' plaintext element).
\item If all checks pass, the parties output $C$.
\end{enumerate}
\end{Boxfig}

